\begin{document}
\section{Introduction}
The transverse field Ising model is a paradigmatic model in quantum many-body physics, widely used to study quantum phase transitions and magnetism. This project aims to compute the ground state energy and magnetization of the 1D transverse field Ising model using two approaches:

\begin{itemize}
    \item \textbf{Exact Diagonalization (ED):} Directly construct the Hamiltonian matrix and solve for eigenvalues and eigenstates.
    \item \textbf{Quantum Neural Network Variational Method (QNN-VQE):} Use Qiskit's EstimatorQNN and a physically inspired ansatz, optimizing parameters variationally to approximate the ground state and its observables.
\end{itemize}

\section{Exact Diagonalization Method}
\subsection{Method Overview}
The Hamiltonian is constructed for $N$ spins, including nearest-neighbor $Z_iZ_{i+1}$ coupling and transverse field $X_i$ terms. The matrix is diagonalized using \texttt{scipy.linalg.eigh} to obtain the ground state energy $E_0$ and wavefunction. Magnetization operators in $X$ and $Z$ directions are also constructed, and their expectation values are computed in the ground state.

\subsection{Key Code Structure}
\begin{itemize}
    \item \texttt{build\_hamiltonian(N, J, h)}: Build the Hamiltonian matrix.
    \item \texttt{build\_total\_op(N, op)}: Build total magnetization operators.
    \item \texttt{eigh(H)}: Solve for eigenvalues and eigenvectors.
\end{itemize}

\subsection{Results}
Supports $N=2$ to $10$, outputs $E_0$, $\langle Z \rangle$, $\langle X \rangle$ for each $N$. Visualization of $\langle X \rangle$ as a function of $N$ is also provided.

\section{Quantum Neural Network Variational Method}
\subsection{Method Overview}
Qiskit's EstimatorQNN and TorchConnector are used to combine quantum circuits with PyTorch optimizers. The ansatz is physically inspired: initial RY rotations simulate magnetization direction, followed by layers of RY+RX rotations and nearest-neighbor entanglement. The loss function is the ground state energy, and after optimization, both energy and magnetization are computed.

\subsection{Key Code Structure}
\begin{itemize}
    \item \texttt{build\_ising\_hamiltonian(n, J, h)}: Build the Hamiltonian in Qiskit format.
    \item \texttt{create\_physically\_inspired\_ansatz(n\_qubits, h, J, reps)}: Build the ansatz.
    \item \texttt{EstimatorQNN} and \texttt{TorchConnector}: Interface between quantum circuit and optimizer.
    \item Training loop: AdamW optimizer, 500 steps, loss is energy.
    \item Magnetization is computed as the average over all qubits.
\end{itemize}

\subsection{Results}
Supports $N=2$ to $10$, outputs $E_0$, $\langle Z \rangle$, $\langle X \rangle$ for each $N$. Results are compared with exact diagonalization to verify the effectiveness of the QNN method.

\section{Comparison and Analysis}
Both methods yield ground state energy and magnetization for small systems ($N\leq10$), with QNN results closely matching exact diagonalization. The QNN method is scalable to larger systems and suitable for quantum hardware or simulators. The physically inspired ansatz improves convergence and interpretability.

\section{Conclusion}
This project implements two mainstream numerical approaches for the transverse field Ising model, verifies the accuracy of QNN-VQE for small systems, and lays the foundation for future applications to larger systems or real quantum hardware. Exact diagonalization is suitable for benchmarking, while QNN provides a new tool for quantum many-body research.

%----------------------------------END Ising Model Quantum Report------------------------------------------

\end{document}
%----------------------------------Ising Model Quantum Report------------------------------------------

